\chapter{Outlook}

\section{\label{sec:discussion}Discussion}
Gaining insight into the effect strain induced in crystal and has on the $g$ and $A$ is not a trivial question. In particular the highly anisotropic nature of $YSO$ makes separating the sensitivity to changes in the orientation of $B_{0}$ with respect to the dielectric axis and strain mechanism difficult. For this experimental setup where the sample is hosted in a customised sample holder was designed to allow rotation of the crystal such that experiments could be completed for a variety or orientations. However, the free rotation of the sample hindered the ability to investigate a particular orientation whilst stacking mass on top of the rotation plate. 

Therefore, either the sample rod could be clamped tightly in place to the probe meaning the mass placed on the rotation plate would have no impact on the stress applied to the sample or the connection is loosen and suspected dominant mechanism observed from the sample rotation. It appears in Fig.~\ref{fig:strainexpsecond1} and Fig.~\ref{fig:strainexpthird1} this may be the case as the $\ket{S=1/2,m_{I}=-1/2} \leftrightarrow \ket{S=-1/2,m_{I}=-1/2}$ transition for those angular rotations appears to be fairly insensitivity to crystal rotation with not notable magnetic field resonance shift. Yet the transition shown to be sensitive to small angular rotations in Fig.~\ref{fig:straininducedsplittingeasyspin} undergoes approximately linear shifts. Similarly, measurements of the resonance linewidth in particular for Fig.~\ref{fig:strainexpthird2} appears to show an upwards trend which suggests strain could be the mechanism similarly as it is in Ref.~\citep{PhysRevLett.115.057601}.

Interestingly upon comparison to the simulated EPR spectrum as a function of angular rotation to the experimentally measured EPR peaks in for example Fig.~\ref{fig:strainexpfirstfieldsweep} can be related with good agreement for the $\theta$ = 120.5$\pm 12.4 ^{\circ}$ crystal orientation. Thus each transition undergoes the opposite sign shift to their magnetically degenerate counterpart. The observation is made that the dipole allowed transitions are not shifting in opposite directions as result of the stress applied to the crystal which indicates the strain is producing a reduction in the hyperfine coupling strength.


\section{Conclusions and Future Work}
Despite the experimental results of the investigation of induced shifts of the $g$ and $A$-tensor currently being inconclusive due to the crystal sample rotation during measurement, this research could help to inform future experimental schemes involving YSO. The ability to rotate the sample to different orientations whilst the sample is in the cryostat is beneficial. However, investigation of using a fixed probe would be interesting and where such a setup is not necessarily be trivial to fabricate. The issue is trying to manage the need to secure the sample so that it cannot move but also enable the application of mechanical stress to the sample surface. An alternative optical may be to apply strain using a piezo-actuator such as in \citep{PhysRevLett.115.057601}. Additionally the crystal orientation chosen was dictated by the region where rate of change of the magnetic field resonances as a function of $\theta$ was lowest. However, the broader site I transitions provide larger values of $df/dg$ and would be interesting to investigate next.  

Conversely, the emphasis during measurement on trying to eliminate any crystal rotation may have hindered the ability to effectively measure effects due to applied strain. The clamp connecting the sample holder to the EPR probe may have been preventing the full force due to the masses being applied to the crystal. Additionally, certain transitions were observed to be less sensitive to angular rotations. Therefore, loosening to the friction between the rod and the focusing on the $\ket{S=1/2,m_{I}=-1/2} \leftrightarrow \ket{S=-1/2,m_{I}=-1/2}$ transition may reveal shift resulting from the stress mechanism. Additionally, an initial attempt was made to characterise the effect of small crystal rotations to decouple the transitions from this effect by rotating the crystal in 5-10 $^{\circ}$ increments. However, the sample sensitivity to rotations and the number of transitions in this sample results in many crossings between transitions making it time-consuming and difficult to track for these rotational increments.  

Lastly, another avenue for future exploration is the density function theory calculations used to calculate the stiffness coefficients in Ref.~\citep{Ceramics}. The treatment of the Spin Hamiltonian terms in this model may aid the ability to determined the unknown coefficients of the $\bm{\mathcal{G}}$ and $\bm{\mathcal{A}}$ tensors. 


