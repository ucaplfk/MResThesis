\chapter{Thesis Outline}
\label{ch:Outline}
Below is a description of this thesis outline.
\par
\par
\noindent \textbf{Chapter 1:}{\addtolength{\leftskip}{5 mm} provides a brief introduction to quantum computing followed by a review of recent experimental achievements relating to RE doped YSO and its potential to provide the memory and micro-to-optical conversion in a hybrid quantum computing architecture. Additionally, the applications and investigation of strain effects observed in a hybrid devices when superconducting resonators interface with spin systems are considered. \\*

}

\noindent \textbf{Chapter 2:} {\addtolength{\leftskip}{5 mm}
Section~\ref{sec:electronparamagneticresonance} provides an introduction to electron paramagnetic resonance (EPR) and the spin Hamiltonian. In particular pulsed-ESR is described using the magnetisation vector picture as a semi-classical analogy of a spin-1/2 particle's evolution on the Bloch sphere in the presence of a driving field. In Section~\ref{sec:YSO} the crystalline structure of yttrium orthosilicate (YSO) is presented, followed by an introduction to the electronic structure of rare-earth doped ions. Lastly in Section~\ref{sec:YSO} the properties specific to $^{171}$Yb$^{3+}$ doped YSO are discussed. 
Section~\ref{sec:strain} presents the mathematical relationship between stress and strain. In addition the stiffness matrix components for YSO with derivation of the transformation to conventional dielectric axes is presented. \\*


}

\noindent \textbf{Chapter 3:} {\addtolength{\leftskip}{5 mm} This chapter details the experimental methods relevant for the investigation of strain in rare-earth doped YSO. Section~\ref{sec:EPRexperimentsetup} describes the experimental setup required to complete EPR spectroscopy. Details of the operation of the sample cooling, resonator and spectrometer are presented. Furthermore, fabrication of the sample holders specific to each orientation of the Yb doped YSO sample is given in this section. The simulations completed using the MATLAB toolbox EasySpin are presented in Section~\ref{sec:simulation}. These simulations provide a guide to the orientation of the crystal based on the resonance magnetic field obtained and additionally provide primary investigation of the response to strain for this anisotropic system. \\*


}

\noindent \textbf{Chapter 4:} {\addtolength{\leftskip}{5 mm} Experimental results obtained using EPR spectroscopy and consideration of the effect of applying uniaxial stress to the crystal. \\*


}
\noindent \textbf{Chapter 5:} {\addtolength{\leftskip}{5 mm} This chapter summarises and discusses the results obtained for this project. Additionally possible routes of future work are presented. \\*


}


